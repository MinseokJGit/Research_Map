


Title:

Programming Language-Based Explainable Graph Machine Learning & 
Context Tunneling for Effective Static Program Analysis


Abstract:

In this talk, I will introduce my research on programming language-based inherently explainable graph machine learning and context tunneling, a novel technique that makes static program analysis precise and scalable.
<!--  -->






본 세미나에서는 프로그래밍 언어 기술을 활용하여 개발한 설명 가능한 그래프 기계학습 방법을 소개하고자 합니다. 현재 그래프 데이터는 분자구조, 소셜 네트워크, 전자 상거래, 생물정보학, 프로그램 표현 등 다양한 분야에서 활용되고 있는 중요한 데이터 구조이고, 그래프 기계학습은 그래프 데이터를 분석하는 데 중요한 역할을 합니다. 현재 가장 널리 사용되고 있는 그래프 기계학습 방법은 그래프 신경망(Graph Neural Networks, GNNs)입니다. 그래프 데이터 기계학습에서 그래프 신경망은 높은 정확도를 보이기에 널리 사용되어 왔지만, 왜 모델이 그런 예측을 내리는지에 대한 설명을 제공하지 못한다는 고질적인 문제 또한 가지고 있습니다. 이를 해결하기 위해 다양한 인공신경망 설명 방법들이 제안되어 왔지만, 이들은 모두 두 가지 근본적인 한계를 가지고 있습니다. 첫 번째 한계는 설명을 제공하는 데 추가적인 시간이 필수적이라는 것이고, 두 번째 한계는 제공한 설명이 옳은 설명임을 보장하지 못한다는 것입니다. 인공신경망은 내부 동작을 이해할 수 없는 블랙박스로 취급되기 때문에, 인공신경망을 기반으로 하는 분류 모델을 사용하는 한 이 두 한계점을 극복하는 것은 매우 어려운 일입니다. 이를 근본적으로 해결하기 위해 본 연구에서는 프로그래밍 언어 기술을 활용해 본질적으로 설명가능한 그래프 기계학습 방법을 개발하였고 2가지 핵심 기술들로 구성되어 있습니다. 첫 번째 기술은 그래프 기계학습을 위해 디자인한 특화 프로그래밍 언어(Domain Specific Language)입니다. 본 연구에서 그래프 기계학습 모델은 개발한 특화 프로그래밍 언어로 구현한 프로그램입니다. 모델이 명확한 소스코드를 가지고 있기 때문에 이를 통해 모델의 내부 동작을 명확하게 이해할 수 있습니다. 두 번째 기술은 프로그램 합성 기술입니다. 본 연구에서 기계학습은 도메인 특화 프로그래밍 언어로 작성된 프로그램을 합성해 내는 것으로 정의되고, 개발한 프로그램 합성 기술은 학습 데이터로부터 모델(프로그램)을 자동으로 합성해 내는 기술입니다. 실험에서는 본 연구에서 개발한 방법이 기존의 그래프 기계학습 방법들과 비교해 뒤지지 않은 정확도를 보이면서도 매우 높은 설명력을 가지고 있음을 보였습니다. 자세한 내용과 구체적인 실험 결과들은 세미나에서 소개하도록 하겠습니다.