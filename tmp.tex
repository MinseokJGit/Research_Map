



### 세부 내용.

본 연구는 프로그래밍 언어 기술을 활용하여 본질적으로 설명 가능한 새로운 그래프 기계학습 방법을 개발한 연구임. 
현재 그래프 데이터는 분자구조, 소셜 네트워크, 전자 상거래, 생물정보학, 
프로그램 표현 등 다양한 분야에서 활용되고 있는 중요한 데이터 구조이고, 그래프 기계학습은 그래프 데이터를 분석하는 데 중요한 역할을 함.
현재 그래프 데이터 기계학습에서는 그래프 인공 신경망(Graph Neural Networks)이 높은 정확도를 이유로 널리 사용되어 왔음. 
하지만, 인공신경망 기반 방법들은 내놓은 예측 결과들을 설명하지 못한다는 고질적인 문제점 또한 가지고 있음.
이를 해결하기 위해 다양한 인공신경망 설명 방법들이 제안되어 왔지만, 이들은 모두 두 가지 근본적인 한계를 가지고 있음.
첫 번째 한계는 설명을 제공하는 데 추가적인 시간이 필수적이라는 것이고, 두 번째 한계는 제공한 설명이 옳은 설명임을 보장하지 못한다는 것임. 
인공신경망은 내부 동작을 이해할 수 없는 블랙박스로 취급되기 때문에, 인공신경망을 기반으로 하는 분류 모델을 사용하는 한 이 두 한계점을 극복하는 것은 매우 어려운 일임.
이를 근본적으로 해결하기 위해 본 연구에서는 프로그래밍 언어 기술을 활용해 본질적으로 설명가능한 그래프 기계학습 방법을 개발하였고 2가지 핵심 기술들로 구성되어 있음.
첫 번째 핵심 기술은 그래프 기계학습을 위해 디자인한 특화 프로그래밍 언어(Domain Specific Language)임. 본 연구에서 개발한 그래프 기계학습 모델은 개발한 특화 프로그래밍 언어로 구현한 프로그램이고, 명확한 소스코드를 가지고 있기에 모델의 내부 동작을 명확하게 이해할 수 있음. 
두 번째 핵심 기술은 프로그램 합성 기술임.
본 연구에서 기계학습은 도메인 특화 프로그래밍 언어로 작성된 프로그램을 합성해 내는 것으로 정의되고, 개발한 프로그램 합성 기술은 학습 데이터로부터 모델(프로그램)을 자동으로 합성해 냄.
실험에서는 본 연구에서 개발한 방법이 기존의 그래프 기계학습 방법들과 비교해 뒤지지 않은 정확도를 보이지만 높은 품질의 (간단하면서 옳음이 보장되는) 설명을 제공함을 보였음.




### 기대성과 및 파급효과

본 연구는 (본질적으로 설명 가능한) 새로운 기계학습 방법을 만들어냈다는 점에서 매우 혁신적인 연구임.
현재 그래프 기계학습 방법이 분야를 가지리 않고 널리 사용됨이 따라 설명에 대한 수요도 매우 커지고 있음.
현재 널리 사용되고 있는 그래프 인공 신경망 기반 방법은 내부의 동작을 이해할 수 없는 black-box 모델임.
이러한 상황에서 본 연구는 모델이 사람이 이해할 수 있는 소스코드를 가지게 하는 본질적으로 설명 가능한 그래프 기계학습 방법을 제시하였음.
본질적으로 설명 가능하면서도 기존의 인공신경망 기반 방법들만큼 높은 정확도를 보인 것은 매우 혁신적인 결과임.
본 연구에서 개발한 방법을 원천기술로 사용하여 다양한 그래프 기계학습 문제에 해결책을 제시할 수 있을 것으로 기대됨.
현재 본 연구방향인 "프로그래밍 언어 기반 설명 가능한 그래프 기계학습 방법"은 국가연구개발과제 (세종과학 펠로우쉽)에 선정되어 다양한 후속연구들을 진행중에 있음.





