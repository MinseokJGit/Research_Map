


Overall, the response is fine. But, the basic issue is that the paper is not that well written, and we have to take it on face value that they will adequately improve it. Personally, my opinion is that the paper is too far over the line for us to make the assumption they can get it up to a good standard. There is just way too much confusion in the presentation. They have a lot of work to do to address this.

If this were a round 1 submission, then I would be recommending they could resubmit in round 2. But, we've missed the boat on that.





RS-2024-00333885 (협약용)연구개발계획서 검토완료 안내

1. 귀 기관의 무궁한 발전을 기원합니다.
2. 귀 기관에서 신청하신 RS-2024-00333885 과제의 (협약용)연구개발계획서 검토결과를 아래와 같이 안내드립니다.

    - 아 래 -  
가. 전문기관               : 한국연구재단
나. 세부사업명            : 개인기초연구(과기정통부)(R&D)
다. 연구개발과제번호   : RS-2024-00333885  
라. 연구개발과제명      : 설명 가능한 그래프 기계학습 방법 개발을 위한 프로그래밍 언어 기술 연구  
마. 주관연구개발기관   : 고려대학교
바. 협약진행상태         : 협약체결
사. 검토완료인 대상과제 협약신청 절차
- (PC)범부처통합연구지원시스템(http://www.iris.go.kr)>>로그인>>업무포털>>과제수행>>협약신청 메뉴 이동
- 연구개발과제번호 검색 후 협약체결 진행

감사합니다.



